
% Default to the notebook output style

    


% Inherit from the specified cell style.




    
\documentclass[11pt]{article}

    
    
    \usepackage[T1]{fontenc}
    % Nicer default font (+ math font) than Computer Modern for most use cases
    \usepackage{mathpazo}

    % Basic figure setup, for now with no caption control since it's done
    % automatically by Pandoc (which extracts ![](path) syntax from Markdown).
    \usepackage{graphicx}
    % We will generate all images so they have a width \maxwidth. This means
    % that they will get their normal width if they fit onto the page, but
    % are scaled down if they would overflow the margins.
    \makeatletter
    \def\maxwidth{\ifdim\Gin@nat@width>\linewidth\linewidth
    \else\Gin@nat@width\fi}
    \makeatother
    \let\Oldincludegraphics\includegraphics
    % Set max figure width to be 80% of text width, for now hardcoded.
    \renewcommand{\includegraphics}[1]{\Oldincludegraphics[width=.8\maxwidth]{#1}}
    % Ensure that by default, figures have no caption (until we provide a
    % proper Figure object with a Caption API and a way to capture that
    % in the conversion process - todo).
    \usepackage{caption}
    \DeclareCaptionLabelFormat{nolabel}{}
    \captionsetup{labelformat=nolabel}

    \usepackage{adjustbox} % Used to constrain images to a maximum size 
    \usepackage{xcolor} % Allow colors to be defined
    \usepackage{enumerate} % Needed for markdown enumerations to work
    \usepackage{geometry} % Used to adjust the document margins
    \usepackage{amsmath} % Equations
    \usepackage{amssymb} % Equations
    \usepackage{textcomp} % defines textquotesingle
    % Hack from http://tex.stackexchange.com/a/47451/13684:
    \AtBeginDocument{%
        \def\PYZsq{\textquotesingle}% Upright quotes in Pygmentized code
    }
    \usepackage{upquote} % Upright quotes for verbatim code
    \usepackage{eurosym} % defines \euro
    \usepackage[mathletters]{ucs} % Extended unicode (utf-8) support
    \usepackage[utf8x]{inputenc} % Allow utf-8 characters in the tex document
    \usepackage{fancyvrb} % verbatim replacement that allows latex
    \usepackage{grffile} % extends the file name processing of package graphics 
                         % to support a larger range 
    % The hyperref package gives us a pdf with properly built
    % internal navigation ('pdf bookmarks' for the table of contents,
    % internal cross-reference links, web links for URLs, etc.)
    \usepackage{hyperref}
    \usepackage{longtable} % longtable support required by pandoc >1.10
    \usepackage{booktabs}  % table support for pandoc > 1.12.2
    \usepackage[inline]{enumitem} % IRkernel/repr support (it uses the enumerate* environment)
    \usepackage[normalem]{ulem} % ulem is needed to support strikethroughs (\sout)
                                % normalem makes italics be italics, not underlines
    

    
    
    % Colors for the hyperref package
    \definecolor{urlcolor}{rgb}{0,.145,.698}
    \definecolor{linkcolor}{rgb}{.71,0.21,0.01}
    \definecolor{citecolor}{rgb}{.12,.54,.11}

    % ANSI colors
    \definecolor{ansi-black}{HTML}{3E424D}
    \definecolor{ansi-black-intense}{HTML}{282C36}
    \definecolor{ansi-red}{HTML}{E75C58}
    \definecolor{ansi-red-intense}{HTML}{B22B31}
    \definecolor{ansi-green}{HTML}{00A250}
    \definecolor{ansi-green-intense}{HTML}{007427}
    \definecolor{ansi-yellow}{HTML}{DDB62B}
    \definecolor{ansi-yellow-intense}{HTML}{B27D12}
    \definecolor{ansi-blue}{HTML}{208FFB}
    \definecolor{ansi-blue-intense}{HTML}{0065CA}
    \definecolor{ansi-magenta}{HTML}{D160C4}
    \definecolor{ansi-magenta-intense}{HTML}{A03196}
    \definecolor{ansi-cyan}{HTML}{60C6C8}
    \definecolor{ansi-cyan-intense}{HTML}{258F8F}
    \definecolor{ansi-white}{HTML}{C5C1B4}
    \definecolor{ansi-white-intense}{HTML}{A1A6B2}

    % commands and environments needed by pandoc snippets
    % extracted from the output of `pandoc -s`
    \providecommand{\tightlist}{%
      \setlength{\itemsep}{0pt}\setlength{\parskip}{0pt}}
    \DefineVerbatimEnvironment{Highlighting}{Verbatim}{commandchars=\\\{\}}
    % Add ',fontsize=\small' for more characters per line
    \newenvironment{Shaded}{}{}
    \newcommand{\KeywordTok}[1]{\textcolor[rgb]{0.00,0.44,0.13}{\textbf{{#1}}}}
    \newcommand{\DataTypeTok}[1]{\textcolor[rgb]{0.56,0.13,0.00}{{#1}}}
    \newcommand{\DecValTok}[1]{\textcolor[rgb]{0.25,0.63,0.44}{{#1}}}
    \newcommand{\BaseNTok}[1]{\textcolor[rgb]{0.25,0.63,0.44}{{#1}}}
    \newcommand{\FloatTok}[1]{\textcolor[rgb]{0.25,0.63,0.44}{{#1}}}
    \newcommand{\CharTok}[1]{\textcolor[rgb]{0.25,0.44,0.63}{{#1}}}
    \newcommand{\StringTok}[1]{\textcolor[rgb]{0.25,0.44,0.63}{{#1}}}
    \newcommand{\CommentTok}[1]{\textcolor[rgb]{0.38,0.63,0.69}{\textit{{#1}}}}
    \newcommand{\OtherTok}[1]{\textcolor[rgb]{0.00,0.44,0.13}{{#1}}}
    \newcommand{\AlertTok}[1]{\textcolor[rgb]{1.00,0.00,0.00}{\textbf{{#1}}}}
    \newcommand{\FunctionTok}[1]{\textcolor[rgb]{0.02,0.16,0.49}{{#1}}}
    \newcommand{\RegionMarkerTok}[1]{{#1}}
    \newcommand{\ErrorTok}[1]{\textcolor[rgb]{1.00,0.00,0.00}{\textbf{{#1}}}}
    \newcommand{\NormalTok}[1]{{#1}}
    
    % Additional commands for more recent versions of Pandoc
    \newcommand{\ConstantTok}[1]{\textcolor[rgb]{0.53,0.00,0.00}{{#1}}}
    \newcommand{\SpecialCharTok}[1]{\textcolor[rgb]{0.25,0.44,0.63}{{#1}}}
    \newcommand{\VerbatimStringTok}[1]{\textcolor[rgb]{0.25,0.44,0.63}{{#1}}}
    \newcommand{\SpecialStringTok}[1]{\textcolor[rgb]{0.73,0.40,0.53}{{#1}}}
    \newcommand{\ImportTok}[1]{{#1}}
    \newcommand{\DocumentationTok}[1]{\textcolor[rgb]{0.73,0.13,0.13}{\textit{{#1}}}}
    \newcommand{\AnnotationTok}[1]{\textcolor[rgb]{0.38,0.63,0.69}{\textbf{\textit{{#1}}}}}
    \newcommand{\CommentVarTok}[1]{\textcolor[rgb]{0.38,0.63,0.69}{\textbf{\textit{{#1}}}}}
    \newcommand{\VariableTok}[1]{\textcolor[rgb]{0.10,0.09,0.49}{{#1}}}
    \newcommand{\ControlFlowTok}[1]{\textcolor[rgb]{0.00,0.44,0.13}{\textbf{{#1}}}}
    \newcommand{\OperatorTok}[1]{\textcolor[rgb]{0.40,0.40,0.40}{{#1}}}
    \newcommand{\BuiltInTok}[1]{{#1}}
    \newcommand{\ExtensionTok}[1]{{#1}}
    \newcommand{\PreprocessorTok}[1]{\textcolor[rgb]{0.74,0.48,0.00}{{#1}}}
    \newcommand{\AttributeTok}[1]{\textcolor[rgb]{0.49,0.56,0.16}{{#1}}}
    \newcommand{\InformationTok}[1]{\textcolor[rgb]{0.38,0.63,0.69}{\textbf{\textit{{#1}}}}}
    \newcommand{\WarningTok}[1]{\textcolor[rgb]{0.38,0.63,0.69}{\textbf{\textit{{#1}}}}}
    
    
    % Define a nice break command that doesn't care if a line doesn't already
    % exist.
    \def\br{\hspace*{\fill} \\* }
    % Math Jax compatability definitions
    \def\gt{>}
    \def\lt{<}
    % Document parameters
    \title{Sentiment\_Mining}
    
    
    

    % Pygments definitions
    
\makeatletter
\def\PY@reset{\let\PY@it=\relax \let\PY@bf=\relax%
    \let\PY@ul=\relax \let\PY@tc=\relax%
    \let\PY@bc=\relax \let\PY@ff=\relax}
\def\PY@tok#1{\csname PY@tok@#1\endcsname}
\def\PY@toks#1+{\ifx\relax#1\empty\else%
    \PY@tok{#1}\expandafter\PY@toks\fi}
\def\PY@do#1{\PY@bc{\PY@tc{\PY@ul{%
    \PY@it{\PY@bf{\PY@ff{#1}}}}}}}
\def\PY#1#2{\PY@reset\PY@toks#1+\relax+\PY@do{#2}}

\expandafter\def\csname PY@tok@w\endcsname{\def\PY@tc##1{\textcolor[rgb]{0.73,0.73,0.73}{##1}}}
\expandafter\def\csname PY@tok@c\endcsname{\let\PY@it=\textit\def\PY@tc##1{\textcolor[rgb]{0.25,0.50,0.50}{##1}}}
\expandafter\def\csname PY@tok@cp\endcsname{\def\PY@tc##1{\textcolor[rgb]{0.74,0.48,0.00}{##1}}}
\expandafter\def\csname PY@tok@k\endcsname{\let\PY@bf=\textbf\def\PY@tc##1{\textcolor[rgb]{0.00,0.50,0.00}{##1}}}
\expandafter\def\csname PY@tok@kp\endcsname{\def\PY@tc##1{\textcolor[rgb]{0.00,0.50,0.00}{##1}}}
\expandafter\def\csname PY@tok@kt\endcsname{\def\PY@tc##1{\textcolor[rgb]{0.69,0.00,0.25}{##1}}}
\expandafter\def\csname PY@tok@o\endcsname{\def\PY@tc##1{\textcolor[rgb]{0.40,0.40,0.40}{##1}}}
\expandafter\def\csname PY@tok@ow\endcsname{\let\PY@bf=\textbf\def\PY@tc##1{\textcolor[rgb]{0.67,0.13,1.00}{##1}}}
\expandafter\def\csname PY@tok@nb\endcsname{\def\PY@tc##1{\textcolor[rgb]{0.00,0.50,0.00}{##1}}}
\expandafter\def\csname PY@tok@nf\endcsname{\def\PY@tc##1{\textcolor[rgb]{0.00,0.00,1.00}{##1}}}
\expandafter\def\csname PY@tok@nc\endcsname{\let\PY@bf=\textbf\def\PY@tc##1{\textcolor[rgb]{0.00,0.00,1.00}{##1}}}
\expandafter\def\csname PY@tok@nn\endcsname{\let\PY@bf=\textbf\def\PY@tc##1{\textcolor[rgb]{0.00,0.00,1.00}{##1}}}
\expandafter\def\csname PY@tok@ne\endcsname{\let\PY@bf=\textbf\def\PY@tc##1{\textcolor[rgb]{0.82,0.25,0.23}{##1}}}
\expandafter\def\csname PY@tok@nv\endcsname{\def\PY@tc##1{\textcolor[rgb]{0.10,0.09,0.49}{##1}}}
\expandafter\def\csname PY@tok@no\endcsname{\def\PY@tc##1{\textcolor[rgb]{0.53,0.00,0.00}{##1}}}
\expandafter\def\csname PY@tok@nl\endcsname{\def\PY@tc##1{\textcolor[rgb]{0.63,0.63,0.00}{##1}}}
\expandafter\def\csname PY@tok@ni\endcsname{\let\PY@bf=\textbf\def\PY@tc##1{\textcolor[rgb]{0.60,0.60,0.60}{##1}}}
\expandafter\def\csname PY@tok@na\endcsname{\def\PY@tc##1{\textcolor[rgb]{0.49,0.56,0.16}{##1}}}
\expandafter\def\csname PY@tok@nt\endcsname{\let\PY@bf=\textbf\def\PY@tc##1{\textcolor[rgb]{0.00,0.50,0.00}{##1}}}
\expandafter\def\csname PY@tok@nd\endcsname{\def\PY@tc##1{\textcolor[rgb]{0.67,0.13,1.00}{##1}}}
\expandafter\def\csname PY@tok@s\endcsname{\def\PY@tc##1{\textcolor[rgb]{0.73,0.13,0.13}{##1}}}
\expandafter\def\csname PY@tok@sd\endcsname{\let\PY@it=\textit\def\PY@tc##1{\textcolor[rgb]{0.73,0.13,0.13}{##1}}}
\expandafter\def\csname PY@tok@si\endcsname{\let\PY@bf=\textbf\def\PY@tc##1{\textcolor[rgb]{0.73,0.40,0.53}{##1}}}
\expandafter\def\csname PY@tok@se\endcsname{\let\PY@bf=\textbf\def\PY@tc##1{\textcolor[rgb]{0.73,0.40,0.13}{##1}}}
\expandafter\def\csname PY@tok@sr\endcsname{\def\PY@tc##1{\textcolor[rgb]{0.73,0.40,0.53}{##1}}}
\expandafter\def\csname PY@tok@ss\endcsname{\def\PY@tc##1{\textcolor[rgb]{0.10,0.09,0.49}{##1}}}
\expandafter\def\csname PY@tok@sx\endcsname{\def\PY@tc##1{\textcolor[rgb]{0.00,0.50,0.00}{##1}}}
\expandafter\def\csname PY@tok@m\endcsname{\def\PY@tc##1{\textcolor[rgb]{0.40,0.40,0.40}{##1}}}
\expandafter\def\csname PY@tok@gh\endcsname{\let\PY@bf=\textbf\def\PY@tc##1{\textcolor[rgb]{0.00,0.00,0.50}{##1}}}
\expandafter\def\csname PY@tok@gu\endcsname{\let\PY@bf=\textbf\def\PY@tc##1{\textcolor[rgb]{0.50,0.00,0.50}{##1}}}
\expandafter\def\csname PY@tok@gd\endcsname{\def\PY@tc##1{\textcolor[rgb]{0.63,0.00,0.00}{##1}}}
\expandafter\def\csname PY@tok@gi\endcsname{\def\PY@tc##1{\textcolor[rgb]{0.00,0.63,0.00}{##1}}}
\expandafter\def\csname PY@tok@gr\endcsname{\def\PY@tc##1{\textcolor[rgb]{1.00,0.00,0.00}{##1}}}
\expandafter\def\csname PY@tok@ge\endcsname{\let\PY@it=\textit}
\expandafter\def\csname PY@tok@gs\endcsname{\let\PY@bf=\textbf}
\expandafter\def\csname PY@tok@gp\endcsname{\let\PY@bf=\textbf\def\PY@tc##1{\textcolor[rgb]{0.00,0.00,0.50}{##1}}}
\expandafter\def\csname PY@tok@go\endcsname{\def\PY@tc##1{\textcolor[rgb]{0.53,0.53,0.53}{##1}}}
\expandafter\def\csname PY@tok@gt\endcsname{\def\PY@tc##1{\textcolor[rgb]{0.00,0.27,0.87}{##1}}}
\expandafter\def\csname PY@tok@err\endcsname{\def\PY@bc##1{\setlength{\fboxsep}{0pt}\fcolorbox[rgb]{1.00,0.00,0.00}{1,1,1}{\strut ##1}}}
\expandafter\def\csname PY@tok@kc\endcsname{\let\PY@bf=\textbf\def\PY@tc##1{\textcolor[rgb]{0.00,0.50,0.00}{##1}}}
\expandafter\def\csname PY@tok@kd\endcsname{\let\PY@bf=\textbf\def\PY@tc##1{\textcolor[rgb]{0.00,0.50,0.00}{##1}}}
\expandafter\def\csname PY@tok@kn\endcsname{\let\PY@bf=\textbf\def\PY@tc##1{\textcolor[rgb]{0.00,0.50,0.00}{##1}}}
\expandafter\def\csname PY@tok@kr\endcsname{\let\PY@bf=\textbf\def\PY@tc##1{\textcolor[rgb]{0.00,0.50,0.00}{##1}}}
\expandafter\def\csname PY@tok@bp\endcsname{\def\PY@tc##1{\textcolor[rgb]{0.00,0.50,0.00}{##1}}}
\expandafter\def\csname PY@tok@fm\endcsname{\def\PY@tc##1{\textcolor[rgb]{0.00,0.00,1.00}{##1}}}
\expandafter\def\csname PY@tok@vc\endcsname{\def\PY@tc##1{\textcolor[rgb]{0.10,0.09,0.49}{##1}}}
\expandafter\def\csname PY@tok@vg\endcsname{\def\PY@tc##1{\textcolor[rgb]{0.10,0.09,0.49}{##1}}}
\expandafter\def\csname PY@tok@vi\endcsname{\def\PY@tc##1{\textcolor[rgb]{0.10,0.09,0.49}{##1}}}
\expandafter\def\csname PY@tok@vm\endcsname{\def\PY@tc##1{\textcolor[rgb]{0.10,0.09,0.49}{##1}}}
\expandafter\def\csname PY@tok@sa\endcsname{\def\PY@tc##1{\textcolor[rgb]{0.73,0.13,0.13}{##1}}}
\expandafter\def\csname PY@tok@sb\endcsname{\def\PY@tc##1{\textcolor[rgb]{0.73,0.13,0.13}{##1}}}
\expandafter\def\csname PY@tok@sc\endcsname{\def\PY@tc##1{\textcolor[rgb]{0.73,0.13,0.13}{##1}}}
\expandafter\def\csname PY@tok@dl\endcsname{\def\PY@tc##1{\textcolor[rgb]{0.73,0.13,0.13}{##1}}}
\expandafter\def\csname PY@tok@s2\endcsname{\def\PY@tc##1{\textcolor[rgb]{0.73,0.13,0.13}{##1}}}
\expandafter\def\csname PY@tok@sh\endcsname{\def\PY@tc##1{\textcolor[rgb]{0.73,0.13,0.13}{##1}}}
\expandafter\def\csname PY@tok@s1\endcsname{\def\PY@tc##1{\textcolor[rgb]{0.73,0.13,0.13}{##1}}}
\expandafter\def\csname PY@tok@mb\endcsname{\def\PY@tc##1{\textcolor[rgb]{0.40,0.40,0.40}{##1}}}
\expandafter\def\csname PY@tok@mf\endcsname{\def\PY@tc##1{\textcolor[rgb]{0.40,0.40,0.40}{##1}}}
\expandafter\def\csname PY@tok@mh\endcsname{\def\PY@tc##1{\textcolor[rgb]{0.40,0.40,0.40}{##1}}}
\expandafter\def\csname PY@tok@mi\endcsname{\def\PY@tc##1{\textcolor[rgb]{0.40,0.40,0.40}{##1}}}
\expandafter\def\csname PY@tok@il\endcsname{\def\PY@tc##1{\textcolor[rgb]{0.40,0.40,0.40}{##1}}}
\expandafter\def\csname PY@tok@mo\endcsname{\def\PY@tc##1{\textcolor[rgb]{0.40,0.40,0.40}{##1}}}
\expandafter\def\csname PY@tok@ch\endcsname{\let\PY@it=\textit\def\PY@tc##1{\textcolor[rgb]{0.25,0.50,0.50}{##1}}}
\expandafter\def\csname PY@tok@cm\endcsname{\let\PY@it=\textit\def\PY@tc##1{\textcolor[rgb]{0.25,0.50,0.50}{##1}}}
\expandafter\def\csname PY@tok@cpf\endcsname{\let\PY@it=\textit\def\PY@tc##1{\textcolor[rgb]{0.25,0.50,0.50}{##1}}}
\expandafter\def\csname PY@tok@c1\endcsname{\let\PY@it=\textit\def\PY@tc##1{\textcolor[rgb]{0.25,0.50,0.50}{##1}}}
\expandafter\def\csname PY@tok@cs\endcsname{\let\PY@it=\textit\def\PY@tc##1{\textcolor[rgb]{0.25,0.50,0.50}{##1}}}

\def\PYZbs{\char`\\}
\def\PYZus{\char`\_}
\def\PYZob{\char`\{}
\def\PYZcb{\char`\}}
\def\PYZca{\char`\^}
\def\PYZam{\char`\&}
\def\PYZlt{\char`\<}
\def\PYZgt{\char`\>}
\def\PYZsh{\char`\#}
\def\PYZpc{\char`\%}
\def\PYZdl{\char`\$}
\def\PYZhy{\char`\-}
\def\PYZsq{\char`\'}
\def\PYZdq{\char`\"}
\def\PYZti{\char`\~}
% for compatibility with earlier versions
\def\PYZat{@}
\def\PYZlb{[}
\def\PYZrb{]}
\makeatother


    % Exact colors from NB
    \definecolor{incolor}{rgb}{0.0, 0.0, 0.5}
    \definecolor{outcolor}{rgb}{0.545, 0.0, 0.0}



    
    % Prevent overflowing lines due to hard-to-break entities
    \sloppy 
    % Setup hyperref package
    \hypersetup{
      breaklinks=true,  % so long urls are correctly broken across lines
      colorlinks=true,
      urlcolor=urlcolor,
      linkcolor=linkcolor,
      citecolor=citecolor,
      }
    % Slightly bigger margins than the latex defaults
    
    \geometry{verbose,tmargin=1in,bmargin=1in,lmargin=1in,rmargin=1in}
    
    

    \begin{document}
    
    
    \maketitle
    
    

    
    \#

Sentiment Mining

References: *
http://spark-public.s3.amazonaws.com/nlp/slides/sentiment.pptx *
https://www.cs.uic.edu/\textasciitilde{}liub/FBS/Sentiment-Analysis-tutorial-AAAI-2011.pdf

    \hypertarget{what-is-sentiment-mining}{%
\subsection{1. What is Sentiment Mining
?}\label{what-is-sentiment-mining}}

\begin{itemize}
\tightlist
\item
  Computational study of opinions, sentiments, subjectivity,
  evaluations, attitudes, appraisal, affects, views, emotions, etc.,
  expressed in text, e.g.

  \begin{itemize}
  \tightlist
  \item
    Reviews, blogs, discussions, news, comments, feedback, or any other
    documents
  \item
    See some interesting examples from Liu's AAAI-2011 tutorial (pp
    31-38) source:
    http://spark-public.s3.amazonaws.com/nlp/slides/sentiment.pptx
  \end{itemize}
\end{itemize}

    \hypertarget{a-related-concept-emotion}{%
\subsubsection{1.1. A related concept:
emotion}\label{a-related-concept-emotion}}

\begin{itemize}
\tightlist
\item
  Based on (Parrott, 2001), people have six main emotions,
  \textbf{love}, \textbf{joy}, \textbf{surprise}, \textbf{anger},
  \textbf{sadness}, and \textbf{fear}.\\
\item
  Strengths of opinions/sentiments are related to certain emotions,
  e.g., joy, anger.\\
\item
  However, the concepts of emotions and opinions are not equivalent.
\end{itemize}

    \hypertarget{why-sentiment-mining}{%
\subsection{2. Why sentiment mining}\label{why-sentiment-mining}}

\begin{itemize}
\tightlist
\item
  Our perceptions and beliefs are influnced by others
\item
  Whenever we make decisions, we seek out others' opinion

  \begin{itemize}
  \tightlist
  \item
    Movie: is this review positive or negative?
  \item
    Products: what do people think about the new iPhone?
  \item
    Public sentiment: how is consumer confidence? Is despair increasing?
  \item
    Politics: what do people think about this candidate or issue?
  \item
    Prediction: predict election outcomes or market trends from
    sentiment
  \end{itemize}
\end{itemize}

    \hypertarget{objectives-of-sentiment-mining}{%
\subsection{3. Objectives of Sentiment
Mining}\label{objectives-of-sentiment-mining}}

\begin{itemize}
\tightlist
\item
  Example (from Liu's AAAI-2011 tutorial):

  \begin{itemize}
  \tightlist
  \item
    ID: Abc123 on 5-1-2008 ``I bought an iPhone a few days ago. It is
    such a nice phone. The touch screen is really cool. The voice
    quality is clear too. It is much better than my old Blackberry,
    which was a terrible phone and so difficult to type with its tiny
    keys. However, my mother was mad with me as I did not tell her
    before I bought the phone. She also thought the phone was too
    expensive, \ldots{}''\\
  \end{itemize}
\item
  Elements of sentiment:

  \begin{itemize}
  \tightlist
  \item
    target entity: iPhone, Blackberry
  \item
    Target \textbf{aspect/feature} of attitude:

    \begin{itemize}
    \tightlist
    \item
      iphone: touch screen, voice quality, expensive (price) 
    \item
      Blackberry: tiny keys
    \end{itemize}
  \item
    Type of attitude

    \begin{itemize}
    \tightlist
    \item
      \textbf{positive} or \textbf{negative}: nice phone, terrible phone
    \item
      \textbf{Scale of the attitute}, e.g. {[}1, 5{]}, {[}strongly
      agree, agree, neutral, disagree, strongly disagree{]}
    \end{itemize}
  \item
    Opinion holder: Abc123, my mother
  \item
    Time when the opinion is expressed: 5-1-2008
  \end{itemize}
\end{itemize}

    \hypertarget{sentiment-analysis-tasks}{%
\subsection{4. Sentiment analysis
tasks}\label{sentiment-analysis-tasks}}

Giving a set of text (reviews, documents etc.): 1. Identify objects of
the sentiment analysis * \textbf{Named entities}: company names, brands,
proper names, hashtags etc * Usually object names or synonyms are
explicitly mentioned 2. For each object, identify and extract object
aspects/features that have been commented on in each review text *
\textbf{Explicit} features * e.g.~the \textbf{battery life} of this
camera is too short * \textbf{Implicit} features * e.g.~the camera is
too large (implicit feature: \textbf{size}) 3. Determine whether the
sentiment on the features are positive, negative or neutral. 4. Identify
opinion holder (who) and time 4. Generate a summary of sentiment by
multidimension: - on each feature and on each object - by opinion holder
group and time

    \hypertarget{aspectfeature-detection}{%
\subsection{4.1. Aspect/feature
detection}\label{aspectfeature-detection}}

\begin{itemize}
\tightlist
\item
  \textbf{Explicit features/aspects}: typically can be extracted by
  keywords and synonyms

  \begin{itemize}
  \tightlist
  \item
    Question: how to find synonyms?

    \begin{itemize}
    \tightlist
    \item
      Lexical similarity based on WordNet
      (http://www.nltk.org/howto/wordnet.html)
    \item
      Word vectors
    \end{itemize}
  \item
    Challenge: it may be difficult to find an exhaustive list of
    synonyms for an aspect
  \item
    e.g. Source: Lappas, T., Sabnis, G., \& Valkanas, G. (2016). The
    impact of fake reviews on online visibility: A vulnerability
    assessment of the hotel industry. Information Systems Research,
    27(4), 940-961.
  \end{itemize}
\item
  However, \textbf{implicit features}: may need a \textbf{supervised
  approach} (e.g.~the camera is too large)

  \begin{itemize}
  \tightlist
  \item
    Naive bayes, SVM, CNN with word embedding are perhaps good
    approaches here

    \begin{itemize}
    \tightlist
    \item
      single-label or multi-label classification?
    \end{itemize}
  \item
    Process:

    \begin{itemize}
    \tightlist
    \item
      Select a set of documents with features/aspects both
      explicitly/implicitly mentioned
    \item
      Label each of the documents with features/aspects as classes
    \item
      Train a classification model
    \end{itemize}
  \end{itemize}
\end{itemize}

    \begin{Verbatim}[commandchars=\\\{\}]
{\color{incolor}In [{\color{incolor} }]:} \PY{c+c1}{\PYZsh{} Exercise 4.1.1}
        \PY{k+kn}{from} \PY{n+nn}{IPython}\PY{n+nn}{.}\PY{n+nn}{core}\PY{n+nn}{.}\PY{n+nn}{interactiveshell} \PY{k}{import} \PY{n}{InteractiveShell}
        \PY{n}{InteractiveShell}\PY{o}{.}\PY{n}{ast\PYZus{}node\PYZus{}interactivity} \PY{o}{=} \PY{l+s+s2}{\PYZdq{}}\PY{l+s+s2}{all}\PY{l+s+s2}{\PYZdq{}}
        \PY{k+kn}{from} \PY{n+nn}{nltk}\PY{n+nn}{.}\PY{n+nn}{corpus} \PY{k}{import} \PY{n}{wordnet} \PY{k}{as} \PY{n}{wn}
\end{Verbatim}


    \begin{Verbatim}[commandchars=\\\{\}]
{\color{incolor}In [{\color{incolor} }]:} \PY{c+c1}{\PYZsh{} Get synsets: a collection of synonymous words}
        \PY{n}{wn}\PY{o}{.}\PY{n}{synsets}\PY{p}{(}\PY{l+s+s1}{\PYZsq{}}\PY{l+s+s1}{motorcar}\PY{l+s+s1}{\PYZsq{}}\PY{p}{)}
        
        \PY{c+c1}{\PYZsh{} note motorcar has just one possible context. }
        \PY{c+c1}{\PYZsh{} It is identified by car.n.01}
        \PY{c+c1}{\PYZsh{} the 1st noun (letter n) sense of car}
        \PY{c+c1}{\PYZsh{} \PYZdq{}car\PYZdq{} has different synonyms depending on context}
        
        \PY{c+c1}{\PYZsh{} Show all synonyms under sysnset car.n.1}
        \PY{n}{wn}\PY{o}{.}\PY{n}{synset}\PY{p}{(}\PY{l+s+s1}{\PYZsq{}}\PY{l+s+s1}{car.n.01}\PY{l+s+s1}{\PYZsq{}}\PY{p}{)}\PY{o}{.}\PY{n}{lemma\PYZus{}names}\PY{p}{(}\PY{p}{)}
\end{Verbatim}


    \begin{Verbatim}[commandchars=\\\{\}]
{\color{incolor}In [{\color{incolor} }]:} \PY{k}{for} \PY{n}{synset} \PY{o+ow}{in} \PY{n}{wn}\PY{o}{.}\PY{n}{synsets}\PY{p}{(}\PY{l+s+s1}{\PYZsq{}}\PY{l+s+s1}{car}\PY{l+s+s1}{\PYZsq{}}\PY{p}{)}\PY{p}{:}
            \PY{n+nb}{print}\PY{p}{(}\PY{l+s+s2}{\PYZdq{}}\PY{l+s+se}{\PYZbs{}t}\PY{l+s+s2}{Synset: }\PY{l+s+si}{\PYZob{}\PYZcb{}}\PY{l+s+s2}{\PYZdq{}}\PY{o}{.}\PY{n}{format}\PY{p}{(}\PY{n}{synset}\PY{o}{.}\PY{n}{name}\PY{p}{(}\PY{p}{)}\PY{p}{)}\PY{p}{)}
            \PY{n+nb}{print}\PY{p}{(}\PY{l+s+s2}{\PYZdq{}}\PY{l+s+se}{\PYZbs{}t}\PY{l+s+s2}{Definition: }\PY{l+s+si}{\PYZob{}\PYZcb{}}\PY{l+s+s2}{\PYZdq{}}\PY{o}{.}\PY{n}{format}\PY{p}{(}\PY{n}{synset}\PY{o}{.}\PY{n}{definition}\PY{p}{(}\PY{p}{)}\PY{p}{)}\PY{p}{)}
            \PY{n+nb}{print}\PY{p}{(}\PY{l+s+s2}{\PYZdq{}}\PY{l+s+se}{\PYZbs{}t}\PY{l+s+s2}{Synoymns: }\PY{l+s+si}{\PYZob{}\PYZcb{}}\PY{l+s+se}{\PYZbs{}n}\PY{l+s+s2}{\PYZdq{}}\PY{o}{.}\PY{n}{format}\PY{p}{(}\PY{n}{synset}\PY{o}{.}\PY{n}{lemma\PYZus{}names}\PY{p}{(}\PY{p}{)}\PY{p}{)}\PY{p}{)}
\end{Verbatim}


    \hypertarget{sentiment-detection}{%
\subsection{4.2. Sentiment Detection}\label{sentiment-detection}}

    \hypertarget{challenges-of-sentiment-analysis}{%
\subsubsection{4.2.1. Challenges of sentiment
analysis}\label{challenges-of-sentiment-analysis}}

\begin{itemize}
\tightlist
\item
  Negation:

  \begin{itemize}
  \tightlist
  \item
    e.g., This film should be brilliant. It sounds like a great plot,
    the actors are first grade, and the supporting cast is good as well,
    and Stallone is attempting to deliver a good performance. However,
    it can't hold up.
  \end{itemize}
\item
  Sarcasm and language subtlety: sarcastic sentences are very common in
  political blogs, comments and discussions

  \begin{itemize}
  \tightlist
  \item
    e.g.~This is the kind of movie you go because the theater has
    air-conditioning
  \item
    e.g.~What a great car, it stopped working in the second day
  \item
    e.g.~The top of the picture was much brighter than the bottom
  \end{itemize}
\item
  Domain Dependency

  \begin{itemize}
  \tightlist
  \item
    e.g.~unpredictable movie vs.~unpredictable steering (car domain)
  \end{itemize}
\item
  Lots of emoticons
\end{itemize}

    \hypertarget{unsupervised-sentiment-analysis}{%
\subsubsection{4.2.2. Unsupervised Sentiment
Analysis}\label{unsupervised-sentiment-analysis}}

\begin{itemize}
\tightlist
\item
  Lexicon-based method where sentiment is determined based on
  \textbf{opinion words} (e.g. ``amazing'', ``great'', ``poor'') counted
  near features/aspects.

  \begin{itemize}
  \tightlist
  \item
    Some useful rules:

    \begin{itemize}
    \tightlist
    \item
      \textbf{Negative} sentiment:

      \begin{itemize}
      \tightlist
      \item
        negative words not preceded by a negation within \(n\)
        (e.g.~three) words in the same sentence.
      \item
        positive words preceded by a negation within \(n\) (e.g.~three)
        words in the same sentence.
      \end{itemize}
    \item
      \textbf{Positive} sentiment (in the similar fashion):

      \begin{itemize}
      \tightlist
      \item
        positive words not preceded by a negation within \(n\)
        (e.g.~three) words in the same sentence.
      \item
        negative terms following a negation within \(n\) (e.g.~three)
        words in the same sentence
      \end{itemize}
    \end{itemize}
  \end{itemize}
\item
  \textbf{Polarity}-based (Postive or Negative) approaches:

  \begin{itemize}
  \tightlist
  \item
     WordStat sentiment Dictionary: This is probably one of the largest
    lexicons freely available. It contains \textasciitilde{}14.000 words
    ( 9164 negative and 4847 positive words ) and gives words a binary
    classification (positive or a negative ) score.
  \item
     SentiWordNet; gives the words a positive or negative score between
    0 and 1. It contains about 117.660 words, however only
    \textasciitilde{}29.000 of these words have been scored (either
    positive or negative).
  \item
    LIWC (Linguistic Inquiry and Word Count)(http://www.liwc.net/)
  \item
    Turney Algorithm ( Thumbs Up or Thumbs Down? Semantic Orientation
    Applied to Unsupervised Classification of Reviews)

    \begin{enumerate}
    \def\labelenumi{\arabic{enumi}.}
    \tightlist
    \item
      extract phrases,
    \item
      detect sentiment of phrases

      \begin{itemize}
      \tightlist
      \item
        Use search engine queries to check with cooccurrence of a phrase
        (e.g.~low fees) with ``excellence''/``poor'' (Pointwise Mutual
        Inforamtion)
      \end{itemize}
    \item
      and average the sentiments
    \end{enumerate}
  \end{itemize}
\item
  \textbf{Valence}-based where the \textbf{intensity} of the sentiment
  is considered, e.g.~excellent, good, average

  \begin{itemize}
  \tightlist
  \item
    VADER: A Parsimonious Rule-based Model for Sentiment Analysis of
    Social Media Text 
  \end{itemize}
\end{itemize}

    \hypertarget{vader}{%
\subsubsection{4.2.3. VADER}\label{vader}}

\begin{itemize}
\tightlist
\item
  The method of VADER:

  \begin{enumerate}
  \def\labelenumi{\arabic{enumi}.}
  \tightlist
  \item
    Created lexicons of sentiment-related words (\textasciitilde{}9000)
    - Built based on existing well-established sentiment word-banks
    (e.g.~LIWC). - Incorporated many lexical features

    \begin{itemize}
    \tightlist
    \item
      Western-style emoticons10 (for example, ``:-'')
    \item
      Sentiment-related acronyms (e.g., LOL) and commonly used slangs
      with sentiment value (e.g., ``nah'', ``meh'' and ``giggly'').
    \end{itemize}
  \item
    Rated sentiment-related words were manually rated in terms of
    sentiment intensity through Amazon Mechancical Turk: positive or
    negative (and optionally, to what degree)
  \item
    Implemented heurestics rules:

    \begin{itemize}
    \tightlist
    \item
      \textbf{Punctuation exclamation mark(!)} increases sentiment
      intensity, e.g. \emph{``The food here is good!!!''}
    \item
      \textbf{Capitalization, specifically ALL-CAPS} of a
      sentiment-relevant word increases the sentiment intensity, e.g.
      \emph{``The food here is GREAT!''}
    \item
      \textbf{Degree modifiers} (also called intensifiers,
      e.g.~extremely) increases intensity
    \item
      \textbf{Contrastive conjunction ``but''} signals a shift in
      sentiment polarity, with the sentiment of the text following the
      conjunction being dominant. e.g. \emph{``The food here is great,
      but the service is horrible''}.
    \end{itemize}
  \end{enumerate}
\item
  VADER analyzes a piece of text to see if any of the words in the text
  is present in the lexicon. Sentiment metrics are derived from the
  ratings of such words

  \begin{itemize}
  \tightlist
  \item
    \textbf{Positive}, \textbf{neutral} and \textbf{negative}, represent
    the proportion of the text that falls into those categories.
  \item
    The final metric, the compound score, is the sum of all of the
    lexicon ratings which have been standardized to range between -1 and
    1 based on some heuristics.
  \end{itemize}
\end{itemize}

    \begin{Verbatim}[commandchars=\\\{\}]
{\color{incolor}In [{\color{incolor} }]:} \PY{c+c1}{\PYZsh{} Exercise 4.2.3.1 SentimentIntensityAnalyzer}
        
        \PY{k+kn}{from} \PY{n+nn}{nltk}\PY{n+nn}{.}\PY{n+nn}{sentiment}\PY{n+nn}{.}\PY{n+nn}{vader} \PY{k}{import} \PY{n}{SentimentIntensityAnalyzer}
        
        \PY{n}{sid} \PY{o}{=} \PY{n}{SentimentIntensityAnalyzer}\PY{p}{(}\PY{p}{)}
        
        \PY{n}{text}\PY{o}{=}\PY{l+s+s1}{\PYZsq{}}\PY{l+s+s1}{The food is so good and the atmosphere is nice}\PY{l+s+s1}{\PYZsq{}}
        \PY{n}{ss} \PY{o}{=} \PY{n}{sid}\PY{o}{.}\PY{n}{polarity\PYZus{}scores}\PY{p}{(}\PY{n}{text}\PY{p}{)}
        \PY{n+nb}{print}\PY{p}{(}\PY{n}{ss}\PY{p}{)}
\end{Verbatim}


    \begin{Verbatim}[commandchars=\\\{\}]
{\color{incolor}In [{\color{incolor} }]:} \PY{c+c1}{\PYZsh{} Exercise 4.2.3.2 Easy sentences}
        
        \PY{c+c1}{\PYZsh{}http://www.nltk.org/howto/sentiment.html}
        \PY{c+c1}{\PYZsh{}http://t\PYZhy{}redactyl.io/blog/2017/04/using\PYZhy{}vader\PYZhy{}to\PYZhy{}handle\PYZhy{}sentiment\PYZhy{}analysis\PYZhy{}with\PYZhy{}social\PYZhy{}media\PYZhy{}text.html}
        \PY{c+c1}{\PYZsh{}https://www.researchgate.net/publication/275828927\PYZus{}VADER\PYZus{}A\PYZus{}Parsimonious\PYZus{}Rule\PYZhy{}based\PYZus{}Model\PYZus{}for\PYZus{}Sentiment\PYZus{}Analysis\PYZus{}of\PYZus{}Social\PYZus{}Media\PYZus{}Text}
        
        \PY{k+kn}{from} \PY{n+nn}{nltk}\PY{n+nn}{.}\PY{n+nn}{sentiment}\PY{n+nn}{.}\PY{n+nn}{vader} \PY{k}{import} \PY{n}{SentimentIntensityAnalyzer}
        
        \PY{k+kn}{from} \PY{n+nn}{nltk} \PY{k}{import} \PY{n}{tokenize}
        
        \PY{n}{sentences} \PY{o}{=} \PY{p}{[}\PY{l+s+s2}{\PYZdq{}}\PY{l+s+s2}{VADER is smart, handsome, and funny.}\PY{l+s+s2}{\PYZdq{}}\PY{p}{,} \PY{c+c1}{\PYZsh{} positive sentence example}
         \PY{l+s+s2}{\PYZdq{}}\PY{l+s+s2}{VADER is smart, handsome, and funny!}\PY{l+s+s2}{\PYZdq{}}\PY{p}{,} \PY{c+c1}{\PYZsh{} punctuation emphasis handled correctly (sentiment intensity adjusted)}
         \PY{l+s+s2}{\PYZdq{}}\PY{l+s+s2}{VADER is very smart, handsome, and funny.}\PY{l+s+s2}{\PYZdq{}}\PY{p}{,}  \PY{c+c1}{\PYZsh{} booster words handled correctly (sentiment intensity adjusted)}
         \PY{l+s+s2}{\PYZdq{}}\PY{l+s+s2}{VADER is VERY SMART, handsome, and FUNNY.}\PY{l+s+s2}{\PYZdq{}}\PY{p}{,}  \PY{c+c1}{\PYZsh{} emphasis for ALLCAPS handled}
         \PY{l+s+s2}{\PYZdq{}}\PY{l+s+s2}{VADER is VERY SMART, handsome, and FUNNY!!!}\PY{l+s+s2}{\PYZdq{}}\PY{p}{,}\PY{c+c1}{\PYZsh{} combination of signals \PYZhy{} VADER appropriately adjusts intensity}
         \PY{l+s+s2}{\PYZdq{}}\PY{l+s+s2}{VADER is VERY SMART, really handsome, and INCREDIBLY FUNNY!!!}\PY{l+s+s2}{\PYZdq{}}\PY{p}{,}\PY{c+c1}{\PYZsh{} booster words \PYZam{} punctuation make this close to ceiling for score}
         \PY{l+s+s2}{\PYZdq{}}\PY{l+s+s2}{The book was good.}\PY{l+s+s2}{\PYZdq{}}\PY{p}{,}         \PY{c+c1}{\PYZsh{} positive sentence}
         \PY{l+s+s2}{\PYZdq{}}\PY{l+s+s2}{The book was kind of good.}\PY{l+s+s2}{\PYZdq{}}\PY{p}{,} \PY{c+c1}{\PYZsh{} qualified positive sentence is handled correctly (intensity adjusted)}
         \PY{l+s+s2}{\PYZdq{}}\PY{l+s+s2}{The plot was good, but the characters }\PY{l+s+se}{\PYZbs{}}
        \PY{l+s+s2}{ are uncompelling and the dialog is not great.}\PY{l+s+s2}{\PYZdq{}}\PY{p}{,} \PY{c+c1}{\PYZsh{} mixed negation sentence}
         \PY{l+s+s2}{\PYZdq{}}\PY{l+s+s2}{A really bad, horrible book.}\PY{l+s+s2}{\PYZdq{}}\PY{p}{,}       \PY{c+c1}{\PYZsh{} negative sentence with booster words}
         \PY{l+s+s2}{\PYZdq{}}\PY{l+s+s2}{At least it isn}\PY{l+s+s2}{\PYZsq{}}\PY{l+s+s2}{t a horrible book.}\PY{l+s+s2}{\PYZdq{}}\PY{p}{,} \PY{c+c1}{\PYZsh{} negated negative sentence with contraction}
         \PY{l+s+s2}{\PYZdq{}}\PY{l+s+s2}{:) and :D}\PY{l+s+s2}{\PYZdq{}}     \PY{c+c1}{\PYZsh{} emoticons handled}
         \PY{p}{]}
        
        \PY{c+c1}{\PYZsh{} initalize analyzer}
        
        \PY{n}{sid} \PY{o}{=} \PY{n}{SentimentIntensityAnalyzer}\PY{p}{(}\PY{p}{)}
        
        \PY{k}{for} \PY{n}{sentence} \PY{o+ow}{in} \PY{n}{sentences}\PY{p}{:}
            \PY{n+nb}{print}\PY{p}{(}\PY{n}{sentence}\PY{p}{)}
            \PY{n}{ss} \PY{o}{=} \PY{n}{sid}\PY{o}{.}\PY{n}{polarity\PYZus{}scores}\PY{p}{(}\PY{n}{sentence}\PY{p}{)}
            \PY{k}{for} \PY{n}{k} \PY{o+ow}{in} \PY{n+nb}{sorted}\PY{p}{(}\PY{n}{ss}\PY{p}{)}\PY{p}{:}
                \PY{n+nb}{print}\PY{p}{(}\PY{l+s+s1}{\PYZsq{}}\PY{l+s+si}{\PYZob{}0\PYZcb{}}\PY{l+s+s1}{: }\PY{l+s+si}{\PYZob{}1\PYZcb{}}\PY{l+s+s1}{, }\PY{l+s+s1}{\PYZsq{}}\PY{o}{.}\PY{n}{format}\PY{p}{(}\PY{n}{k}\PY{p}{,} \PY{n}{ss}\PY{p}{[}\PY{n}{k}\PY{p}{]}\PY{p}{)}\PY{p}{)}
            \PY{n+nb}{print}\PY{p}{(}\PY{l+s+s2}{\PYZdq{}}\PY{l+s+se}{\PYZbs{}n}\PY{l+s+s2}{\PYZdq{}}\PY{p}{)}
\end{Verbatim}


    \begin{Verbatim}[commandchars=\\\{\}]
{\color{incolor}In [{\color{incolor} }]:} \PY{c+c1}{\PYZsh{} Exercise 4.2.3.3. Tricky sentences}
        \PY{c+c1}{\PYZsh{} How do you think the performance of VADER}
        \PY{c+c1}{\PYZsh{} for this group of sentences?}
        
        \PY{n}{tricky\PYZus{}sentences} \PY{o}{=} \PY{p}{[}
            \PY{l+s+s2}{\PYZdq{}}\PY{l+s+s2}{Sentiment analysis has never been good.}\PY{l+s+s2}{\PYZdq{}}\PY{p}{,}
            \PY{l+s+s2}{\PYZdq{}}\PY{l+s+s2}{Sentiment analysis with VADER has never been this good.}\PY{l+s+s2}{\PYZdq{}}\PY{p}{,}
            \PY{l+s+s2}{\PYZdq{}}\PY{l+s+s2}{Warren Beatty has never been so entertaining.}\PY{l+s+s2}{\PYZdq{}}\PY{p}{,}
            \PY{l+s+s2}{\PYZdq{}}\PY{l+s+s2}{I won}\PY{l+s+s2}{\PYZsq{}}\PY{l+s+s2}{t say that the movie is astounding and I wouldn}\PY{l+s+s2}{\PYZsq{}}\PY{l+s+s2}{t claim that }\PY{l+s+s2}{\PYZdq{}}\PY{p}{]}
        
        \PY{c+c1}{\PYZsh{} initalize analyzer}
        \PY{n}{sid} \PY{o}{=} \PY{n}{SentimentIntensityAnalyzer}\PY{p}{(}\PY{p}{)}
        
        \PY{k}{for} \PY{n}{sentence} \PY{o+ow}{in} \PY{n}{tricky\PYZus{}sentences}\PY{p}{:}
            \PY{n+nb}{print}\PY{p}{(}\PY{n}{sentence}\PY{p}{)}
            \PY{n}{ss} \PY{o}{=} \PY{n}{sid}\PY{o}{.}\PY{n}{polarity\PYZus{}scores}\PY{p}{(}\PY{n}{sentence}\PY{p}{)}
            \PY{k}{for} \PY{n}{k} \PY{o+ow}{in} \PY{n+nb}{sorted}\PY{p}{(}\PY{n}{ss}\PY{p}{)}\PY{p}{:}
                \PY{n+nb}{print}\PY{p}{(}\PY{l+s+s1}{\PYZsq{}}\PY{l+s+si}{\PYZob{}0\PYZcb{}}\PY{l+s+s1}{: }\PY{l+s+si}{\PYZob{}1\PYZcb{}}\PY{l+s+s1}{, }\PY{l+s+s1}{\PYZsq{}}\PY{o}{.}\PY{n}{format}\PY{p}{(}\PY{n}{k}\PY{p}{,} \PY{n}{ss}\PY{p}{[}\PY{n}{k}\PY{p}{]}\PY{p}{)}\PY{p}{)}
            \PY{n+nb}{print}\PY{p}{(}\PY{l+s+s2}{\PYZdq{}}\PY{l+s+se}{\PYZbs{}n}\PY{l+s+s2}{\PYZdq{}}\PY{p}{)}
\end{Verbatim}


    \begin{Verbatim}[commandchars=\\\{\}]
{\color{incolor}In [{\color{incolor} }]:} \PY{c+c1}{\PYZsh{} Exercise 4.2.3.4. Tricky Paragraph}
        
        \PY{c+c1}{\PYZsh{} Deal with Paragraph}
        \PY{c+c1}{\PYZsh{} question: if a paragraph contains mixed positive and }
        \PY{c+c1}{\PYZsh{} negative sentences, how do you determine the sentiment}
        \PY{c+c1}{\PYZsh{} of the entire paragraph?}
        
        \PY{n}{paragraph} \PY{o}{=} \PY{l+s+s2}{\PYZdq{}}\PY{l+s+s2}{This film should be brilliant. }\PY{l+s+se}{\PYZbs{}}
        \PY{l+s+s2}{             It sounds like a great plot, the actors are first grade, }\PY{l+s+se}{\PYZbs{}}
        \PY{l+s+s2}{             and the supporting cast is good as well, }\PY{l+s+se}{\PYZbs{}}
        \PY{l+s+s2}{             and Stallone is attempting to deliver a good performance. }\PY{l+s+se}{\PYZbs{}}
        \PY{l+s+s2}{             However, it can’t hold up.}\PY{l+s+s2}{\PYZdq{}}
        
        \PY{c+c1}{\PYZsh{} split into sentences}
        \PY{n}{lines\PYZus{}list} \PY{o}{=} \PY{n}{tokenize}\PY{o}{.}\PY{n}{sent\PYZus{}tokenize}\PY{p}{(}\PY{n}{paragraph}\PY{p}{)}
        
        \PY{c+c1}{\PYZsh{} initalize analyzer}
        \PY{n}{sid} \PY{o}{=} \PY{n}{SentimentIntensityAnalyzer}\PY{p}{(}\PY{p}{)}
        
        \PY{c+c1}{\PYZsh{} analyze the sentiment sentence by sentence}
        
        \PY{k}{for} \PY{n}{sentence} \PY{o+ow}{in} \PY{n}{lines\PYZus{}list}\PY{p}{:}
            \PY{n+nb}{print}\PY{p}{(}\PY{n}{sentence}\PY{p}{)}
            \PY{n}{ss} \PY{o}{=} \PY{n}{sid}\PY{o}{.}\PY{n}{polarity\PYZus{}scores}\PY{p}{(}\PY{n}{sentence}\PY{p}{)}
            \PY{k}{for} \PY{n}{k} \PY{o+ow}{in} \PY{n+nb}{sorted}\PY{p}{(}\PY{n}{ss}\PY{p}{)}\PY{p}{:}
                \PY{n+nb}{print}\PY{p}{(}\PY{l+s+s1}{\PYZsq{}}\PY{l+s+si}{\PYZob{}0\PYZcb{}}\PY{l+s+s1}{: }\PY{l+s+si}{\PYZob{}1\PYZcb{}}\PY{l+s+s1}{, }\PY{l+s+s1}{\PYZsq{}}\PY{o}{.}\PY{n}{format}\PY{p}{(}\PY{n}{k}\PY{p}{,} \PY{n}{ss}\PY{p}{[}\PY{n}{k}\PY{p}{]}\PY{p}{)}\PY{p}{)}
            \PY{n+nb}{print}\PY{p}{(}\PY{l+s+s2}{\PYZdq{}}\PY{l+s+se}{\PYZbs{}n}\PY{l+s+s2}{\PYZdq{}}\PY{p}{)}
            
        \PY{c+c1}{\PYZsh{} what if you analyze the entire sentence \PYZbs{}}
        \PY{c+c1}{\PYZsh{} as a whole?}
\end{Verbatim}


    \begin{Verbatim}[commandchars=\\\{\}]
{\color{incolor}In [{\color{incolor} }]:} \PY{c+c1}{\PYZsh{} Exercise 4.5. Design a document sentiment classifier based on VADER}
        \PY{c+c1}{\PYZsh{} test your classifier using amazon review dataset}
        \PY{c+c1}{\PYZsh{} and estimate its accuracy}
\end{Verbatim}


    \hypertarget{supervised-sentiment-analysis}{%
\subsubsection{4.2.4 Supervised Sentiment
Analysis}\label{supervised-sentiment-analysis}}

\begin{itemize}
\tightlist
\item
  Naive Bayes (Base line), SVM, CNN.
\item
  Different ways to generate feature space:

  \begin{itemize}
  \tightlist
  \item
    TF-IDF with all tokens
  \item
    with binary counts only
  \item
    Word embedding
  \end{itemize}
\item
  Check lecture notes for ``Text Classification'' and ``Deep Learning
  II''
\end{itemize}


    % Add a bibliography block to the postdoc
    
    
    
    \end{document}
